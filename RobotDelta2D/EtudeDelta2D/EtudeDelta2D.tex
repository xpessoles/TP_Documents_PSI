\documentclass[10pt,fleqn]{article} % Default font size and left-justified equations
\usepackage[%
    pdftitle={TP : Étude du Robot Delta 2D},
    pdfauthor={Xavier Pessoles}]{hyperref}

\input{../../../style/packages}
\input{../../../style/new_style}
\input{../../../style/macros_SII}
\input{../../../style/environment}

\usepackage{amsmath}



\def\discipline{Sciences \\Industrielles de \\ l'Ingénieur}
\def\xxtete{Sciences Industrielles de l'Ingénieur}

\def\classe{PTSI -- PT}
\def\xxnumpartie{Doc. TP}
\def\xxpartie{Étude des systèmes de laboratoire}

\def\xxnumchapitre{Étude du Robot Delta 2D}
\def\xxchapitre{\hspace{.12cm} }

\def\xxposongletx{2}
\def\xxposonglettext{1.45}
\def\xxposonglety{20}
\def\xxonglet{Doc. TP}

\def\xxactivite{Doc. TP}
\def\xxauteur{\textsl{Xavier Pessoles}}

\def\xxcompetences{%
\textsl{%
%\textbf{Savoirs et compétences :}
%\begin{itemize}[label=\ding{112},font=\color{ocre}] 
%\item Mod-C11 : Modélisation géométrique et cinématique des mouvements entre solides indéformables 
%\begin{itemize}[label=\ding{112},font=\color{ocre}] 
%\item Mod-C11.2 : Champ des vecteurs vitesses des points d'un solide
%\end{itemize}
%\end{itemize}
%
}}

\def\xxfigures{
\includegraphics[width=.5\textwidth]{images/maxpid_04}


}%figues de la page de garde
\def\xxpied{%
Étude des systèmes de laboratoire \\
Robot MaxPID%
}

%---------------------------------------------------------------------------


\begin{document}
\chapterimage{images/maxpid_01}
\graphicspath{{../../../style/png/}{images/}}
\input{../../../style/pagegarde_cours}

%\newpage

%%%%%%%%%%%%%%%%%  FIN DE LA COPIE *****************



\section{Paramétrage du robot delta}

On a  :
\begin{itemize}
\item $\vect{OA} = a\vect{x_0}$ et $\vect{OB} =- a\vect{x_0}$  avec $a =\SI{60}{mm}$;
\item $\vect{AD} = \ell\vect{x_1}$ et $\vect{BE} = \ell\vect{x_1'}$ avec $\ell =\SI{170}{mm}$;
\item $\vect{DF} =  L\vect{x_2}$ et $\vect{EF} = L\vect{x_2'}$ avec $L=\SI{350}{mm}$;
\item $\vect{FP}=-b\vect{x_3}-c\vect{y_3}$ avec $b=-\SI{35}{mm}$ et $c=-\SI{75}{mm}$ 
(on pourra montrer que $\vect{x_0}=\vect{x_3}$ et $\vect{y_0}=\vect{y_3}$);
\item $\vect{OF} = x\vect{x_0} +y\vect{y_0} $
\end{itemize}

\section{Modélisation géométrique du Robot Delta 2D}

Le robot delta est un robot à 2 mobilités. Il sera donc nécessaire d'écrire deux fermetures géométriques. 
Commençons pas réaliser la fermeture de la chaîne $O - A - D - F - O$. 

On a donc : 
$\vect{OA}+\vect{AD}+\vect{DF}+\vect{FO} = \vect{0}$, soit 
$a\vect{x_0}+ \ell\vect{x_1}+L\vect{x_2} -x\vect{x_0} -y\vect{y_0} = \vect{0}$.

On projette ensuite dans $\mathcal{B}_{0}$ :
$a\vect{x_0}+ \ell \left( \cos\theta \vect{x_0} + \sin\theta \vect{y_0} \right)+L\left( \cos\psi \vect{x_0} + \sin\psi\vect{y_0} \right) -x\vect{x_0} -y\vect{y_0} = \vect{0}$. 

On alors les expressions suivantes :
$
\left\{
\begin{array}{l}
a+ \ell  \cos\theta +L \cos\psi   -x  = 0 \\
\ell   \sin\theta +L\sin\psi  -y = 0 \\
\end{array}
\right.
$


\end{document}


