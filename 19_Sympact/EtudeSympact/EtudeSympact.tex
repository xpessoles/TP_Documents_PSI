\documentclass[10pt]{article}
\input{style/coursHeadings}
\input{style/programHeadings}
\input{style/macros_SII}
\input{style/macros_Titres}
\input{style/macros_Frames}

%Si le boolen xp est vrai : compilation pour xabi
%Sinon compilation Damien
\newboolean{xp}
\setboolean{xp}{true}

\newboolean{prof}
\setboolean{prof}{true}

\usepackage[%
    pdftitle={Étude complète du Maxpid},
    pdfauthor={Xavier Pessoles},
    colorlinks=true,
    linkcolor=blue,
    citecolor=magenta]{hyperref}

\newif\ifprof
\proftrue
%\proffalse

\newif\iftd
\tdtrue
%\tdfalse

\def\discipline{Sciences Industrielles de l'Ingénieur}
\def\xxtitre{\ifthenelse{\boolean{xp}}{
Étude des systèmes de laboratoire}{
Chapitre  -- }}

\def\xxsoustitre{\ifthenelse{\boolean{xp}}{
Bras robotisé Maxpid}{
Partie  -- }}

\def\xxauteur{\ifthenelse{\boolean{xp}}{
Xavier \textsc{Pessoles}}{}}

\def\xxpied{\ifthenelse{\boolean{xp}}{
\textit{Étude des systèmes de laboratoire}\\
\textit{Barrière Sympact}}{
\xxtitre}}

\def\xxcathegorie{\ifthenelse{\boolean{xp}}{
2020 -- 2021 \\
Xavier \textsc{Pessoles}}{
Informatique - Cours}}





%---------------------------------------------------------------------------


\begin{document}

\ifthenelse{\boolean{xp}}{\input{style/enteteXP}}{\input{style/enteteDI}}

\begin{minipage}[b]{.3\linewidth}
\begin{center}
%\includegraphics[width=.95\linewidth]{images/capsuleuse_ph}

\textit{Système pédagogique}
\end{center}
\end{minipage} \hfill
\begin{minipage}[b]{.3\linewidth}
\begin{center}
%\includegraphics[width=.95\linewidth]{images/capsuleuse_3d}

\textit{Représentation 3D du système}
\end{center}
\end{minipage} \hfill
\begin{minipage}[b]{.3\linewidth}
\begin{center}
%\includegraphics[width=.75\linewidth]{images/CroixMalte_3d}

%\textit{Représentation 3D de la Croix de Malte}
\end{center}
\end{minipage}



\setlength{\parskip}{0ex plus 0.2ex minus 0ex}
 \renewcommand{\contentsname}{}
 \renewcommand{\baselinestretch}{1}

\tableofcontents

 \renewcommand{\baselinestretch}{1.2}
\setlength{\parskip}{2ex plus 0.5ex minus 0.2ex}



\section{Modélisation cinématique de la barrière Sympact}
\subsection{Schéma cinématique}


\begin{center}
 \includegraphics[width=.95\textwidth]{images/fig_01}
\end{center}

\subsection{Détermination de la loi Entrée / Sortie}

La fermeture de chaîne cinématique s'écrit ainsi : 


Projetons cette relation dans le repère $\mathcal{R}_{0}$ :


Pour exprimer la loi entrée sortie, commençons par déterminer $\theta'$ en fonction de $\lambda$ : 





\subsection{Détermination de la loi en vitesse}
%-u'\rsqcine(1-u²)



\subsection{Tracé des courbes} 
Application numérique : 
%\begin{itemize}
%\item $a = 106,3\; \text{mm}$;
%\item $b = 59 \; \text{mm}$;
%\item $c = 70 \; \text{mm}$;
%\item $d = 80 \; \text{mm}$.
%\end{itemize}

%\begin{center}
%\includegraphics[width=.45\textwidth]{images/LoiTheorique}
%
%\textit{Loi Entrée Sortie -- Position angulaire du bras en fonction de la position du moteur} 
%\end{center}

\begin{thebibliography}{2}
\bibitem{xx}{xx}
\end{thebibliography}
\end{document}


